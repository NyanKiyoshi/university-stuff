\section{Les threads sur Android}
On ne bloque jamais l'interface graphique. Toute tâche longue doit s'exécuter dans un thread à part. Les entrées/sorties doivent être faits dans un thread afin d'améliorer la réactivité de l'application.

Il y a donc deux règles :
\begin{enumerate}
    \item Le thread principal (UI thread) ne doit pas être bloqué ;
    \item Il est interdit d’agir sur l’interface utilisateur \textbf{en dehors} du thread principal.
\end{enumerate}

\subsection{Gestion des threads sous Android}
L'API Android fournit trois méthodes pour cela :
\begin{itemize}
    \item \code{Activity.runOnUiThread(Runnable)} ;
    \item \code{View.post(Runnable)} ;
    \item \code{View.postDelayed(Runnable, long)}, permet de retarder une tâche.
\end{itemize}

De plus, elle fournit les classes pour des tâches asynchrones : \code{Handler} et \code{ASyncTask <U, V, W>}.

En plus des classes Java (\code{Thread}, \code{Executor}, \code{TimerTask}, ...).

\noindent \\ \emph{Note : s’il est nécessaire d’agir sur l’interface graphique à partir d’un autre thread, il faut le déléguer au thread principal pour exécution ultérieure.}

\subsection{Création de threads}
On peut créer une classe dérivée de \code{Thread} ou implanter l'interface \code{Runnable}.
\begin{lstlisting}[language=java]
public interface Runnable {
    void run();
}
\end{lstlisting}

On peut soit étendre \code{Thread} dans l'implantation de \code{Runnable}, ou alors, utiliser l'implantation en tant que classe anonyme en la passant à \code{Thread}.

\begin{lstlisting}[language=java]
class myThread extends Thread {
    public void run() {
        // Code qui sera execute par le thread
    }
}

myThread t = new myThread();

t.join();
\end{lstlisting}

\begin{lstlisting}[language=java]
class myTask implement Runnable {
    public void run() {
        // Code qui sera execute par le thread
    }
}

myTask task = new myTask();
Thread t = new Thread(task);

t.join();
\end{lstlisting}

\begin{lstlisting}[language=java]
Thread t = new Thread(new Runnable() {
    public void run() {
        // Code qui sera execute par le thread
    }
});

t.join();
\end{lstlisting}

\subsubsection{Modifier l'UI}
Pour modifier une vue, on va donc procéder de cette manière :
\begin{lstlisting}[language=java]
Handler h = new Handler();
h.postDelayed(new Runnable() {  // postDelayed ou post
    public void run() {
        bt.setText("");
    }
},1000);
\end{lstlisting}


\subsubsection{Tâche périodique}
\begin{lstlisting}[language=java]
Timer t = new Timer();
final Handler h = new Handler();

TimerTask task = new TimerTask() {
    int i = 0;
    public void run() {
        h.post(new Runnable() {
            public void run() {
                bt.setText(i.toString());
            }
        });
        i++;
    }
};

t.schedule(task, 0, 1000);
\end{lstlisting}

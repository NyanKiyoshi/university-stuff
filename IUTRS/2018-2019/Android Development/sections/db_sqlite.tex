\section{Les bases de données SQLite sur Android}
\subsection{Les types}

\begin{itemize}
    \item \code{NULL} ;
    \item \code{INTEGER} (1, 2, 4, 6 ou 8 octets) ;
    \item \code{REAL} (virgule flottante, 8 octets) ;
    \item \code{TEXT} ;
    \item \code{BLOB} (Binary Large OBject).
\end{itemize}

\noindent\\ \textbf{Attention ! Les clés étrangères ne sont pas supportées par ce SGBD.}

\subsection{Création d'une base de données}
Pour créer une base de données, on utilise \code{SQLiteOpenHelper} et on surcharge la méthode \code{onCreate()}.

\begin{lstlisting}[language=java]
public class MyDataBase extends SQLiteOpenHelper {
    // En-têtes obligatoires
    private static final String DATABASE_NAME = "myDatabase.db";
    private static final int DATABASE_VERSION = 2;  // => onUpgrade() si la base n'est à jour.
    
    // Table Names
    private static final String TABLE1_NAME = "myTable1";
    private static final String TABLE1_NAME = "myTable2";

    // Column names
    // ...

    private static final String TABLE1_CREATE =
        "CREATE TABLE " + TABLE1_NAME + " (" +
        KEY_ID + "INTEGER PRIMARY KEY,"+  ...);
        
    // ...
    
    MyDataBase(Context context) {
        super(context, DATABASE_NAME, null, DATABASE_VERSION);
    }
    
    // Appelé si la base n'existe pas
    @Override
    public void onCreate(SQLiteDatabase db) {
        db.execSQL(MYBASE_TABLE_CREATE);
    }
}

MyDataBase mDbHelper = new MyDataBase(getContext());
\end{lstlisting}

\subsection{Requêtes sur une base}
\subsubsection{Insertions}
\begin{lstlisting}[language=java]
// Gets the data repository in write mode
SQLiteDatabase db = mDbHelper.getWritableDatabase();

// Create a new map of values, where column names are the keys
ContentValues values = new ContentValues();
values.put(FeedEntry.COLUMN_NAME_ENTRY_ID, id);
values.put(FeedEntry.COLUMN_NAME_TITLE, title);
values.put(FeedEntry.COLUMN_NAME_CONTENT, content);

// Insert the new row, returning the primary key value of the new row
long newRowId;
newRowId = db.insert(FeedEntry.TABLE_NAME, FeedEntry.COLUMN_NAME_NULLABLE, values);
\end{lstlisting}

\subsubsection{Requêtes}
\begin{lstlisting}[language=java]
SQLiteDatabase db = mDbHelper.getReadableDatabase();

// Define a projection that specifies which columns from the database
// you will actually use after this query.
String[] projection = {
    FeedEntry._ID,
    FeedEntry.COLUMN_NAME_TITLE,
    FeedEntry.COLUMN_NAME_UPDATED,
    // ...
};

// How you want the results sorted in the resulting Cursor
String sortOrder = FeedEntry.COLUMN_NAME_UPDATED + " DESC";
Cursor cursor = db.query(
    // The table to query
    FeedEntry.TABLE_NAME,

    // The columns to return
    projection,

    // The columns for the WHERE clause
    selection,

    // The values for the WHERE clause
    selectionArgs,

    // don't group the rows
    null,

    // don't filter by row groups
    null,

    // The sort order
    sortOrder
);

// Manipulate the cursor
cursor.moveToFirst();

long itemId = cursor.getLong(
    cursor.getColumnIndexOrThrow(FeedEntry._ID));
\end{lstlisting}

\noindent\\ Note : on peut aussi utiliser \code{db.rawQuery(s)} pour
faire une requête en pure SQL (attention aux injections).

\subsubsection{Mises à jour et suppressions}
Même chose que \code{db.insert(...)} mais avec les méthodes \code{db.update(...)} et \code{db.delete(...)}.

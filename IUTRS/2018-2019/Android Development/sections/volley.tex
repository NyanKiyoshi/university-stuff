\section{La gestion de tâches web asynchrones avec la librairie Volley}
La librairie Android \emph{Volley} permet de faire des requêtes HTTP asynchrones de manière très largement simplifiée et bien plus performante.

\subsection{Installation}
Pour installer la librairie Volley, il suffit d'ajouter la dépendance \code{com.android.volley:volley:1.1.1} dans les dépendances du fichier \code{build.gradle} de votre projet ou application (\code{app/build.gradle} par exemple).

\begin{lstlisting}
dependencies {
    ...
    implementation 'com.android.volley:volley:1.1.1'
}
\end{lstlisting}

\subsection{Exemple de requête JSON}
\subsubsection{Ajout de l'autorisation INTERNET dans le manifest}
Afin de connecter votre application à internet, il faut définir la permission dans le manifest. Pour cela, dans \code{AndroidManifest.xml} ajoutez :

\begin{lstlisting}[language=xml]
<?xml version="1.0" encoding="utf-8"?>
<manifest xmlns:android="http://schemas.android.com/apk/res/android"
package="com.example.tp3weather">
    <!-- ... -->
    <uses-permission android:name="android.permission.INTERNET"></uses-permission>
    <!-- ... -->
\end{lstlisting}

\subsubsection{Définition de la queue associée à l'activité courante}
Tout d'abord il faut définir un objet qui contiendra la queue des requêtes HTTP d'une activité ou d'un contexte donné. Pour cela, on ajoutera une propriété de type \code{RequestQueue} en important \code{com.android.volley.RequestQueue} dans notre activité. Puis, on la peuplera en faisant cet appel : \code{Volley.newRequestQueue(this);} lors de la création de l'activité.

\subsubsection{Lancement d'une requête HTTP}
Maintenant qu'on a défini une queue, on peut demander à \emph{Volley} de faire une requête qui attend en réponse un objet JSON (\code{JSONObject} en Java). Pour cela, on fera l'appel suivant : \code{notreQueue.add(new JsonObjectRequest(url, null, listener, errorListener));} (qui est de \code{com.android.volley.toolbox}).

\begin{description}
    \item[\code{listener}] est une fonction qui prend en paramètre un objet de type \code{JSONObject}.
    \item[\code{errorListener}] est une fonction \textbf{facultative} qui prend un objet \code{VolleyError}. Cet objet contient une propriété \code{networkResponse} qui est \code{null} si la requête a échouée, autrement, elle contient la réponse HTTP avec les propriétés suivantes :
        \begin{description}
            \item[\code{statusCode: int}] Le code HTTP.
            \item[\code{notModified: bool}] Vrai si la réponse est une HTTP 304 (Content Not Modified).
            \item[\code{headers: Map<String,String>}] Les en-têtes de la réponses HTTP.
            \item[\code{data: byte[]}] La réponse brute du serveur.
        \end{description}
\end{description}

\subsubsection{Les différents types de requêtes}
La \emph{toolbox} de la librarie Volley définie \href{https://afzaln.com/volley/com/android/volley/toolbox/package-tree.html}{plusieurs méthodes} en plus que celle de l'exemple précédent (JSON). Elles sont les suivantes :

\begin{description}
    \item[\code{StringRequest}] Une simple requête HTTP qui retourne le corps en une chaîne (\code{String}).
    \item[\code{JsonRequest}] Attend et renvoie un \code{JSONObject}.
    \item[\code{ImageRequest}] Attend et renvoie une image sous forme un objet \code{android.graphics.Bitmap}.
\end{description}

\subsubsection{Exemple complet}
\begin{lstlisting}[language=java]
String url = "http://my-json-feed";

JsonObjectRequest jsonObjectRequest = new JsonObjectRequest
        (Request.Method.GET, url, null, new Response.Listener<JSONObject>() {

    @Override
    public void onResponse(JSONObject response) {
        mTextView.setText("Response: " + response.toString());
    }
}, new Response.ErrorListener() {

    @Override
    public void onErrorResponse(VolleyError error) {
        // TODO: Handle error

    }
});

// Access the RequestQueue through your singleton class.
MySingleton.getInstance(this).addToRequestQueue(jsonObjectRequest);
\end{lstlisting}

\subsection{Création de requêtes avec type personnalisé}
On peut facilement définir des requêtes personnalisées, comme par exemple pour un Bitmap :
\begin{lstlisting}[language=java]
import android.graphics.Bitmap;
import android.graphics.BitmapFactory;

import com.android.volley.NetworkResponse;
import com.android.volley.Request;
import com.android.volley.Response;
import com.android.volley.toolbox.HttpHeaderParser;

public class BitmapRequest extends Request<Bitmap> {
    private final Response.Listener<Bitmap> listener;

    public BitmapRequest(
            String url,
            Response.Listener<Bitmap> listener,
            Response.ErrorListener errorListener) {

        super(Method.GET, url, errorListener);
        this.listener = listener;
    }

    @Override
    protected Response<Bitmap> parseNetworkResponse(NetworkResponse response) {
        Bitmap bitmap = BitmapFactory.decodeByteArray(response.data, 0, response.data.length);
        return Response.success(
            bitmap, HttpHeaderParser.parseCacheHeaders(response));
    }

    @Override
    protected void deliverResponse(Bitmap response) {
        listener.onResponse(response);
    }
}
\end{lstlisting}

\section{Comment préparer une réunion ?}
\textbf{Une réunion réussie se prépare.}

\subsection{En tant qu'invité ou participant}
L'invité prend un rôle d'observateur, alors que le participant à un rôle propre.

\subsubsection{Regarder l'ordre du jour}
Bien qu'il leur sera rappelé, les invités et participants doivent avoir lu l'ordre du jour qu'ils ont reçu quelques jours auparavant. Ainsi, ils peuvent savoir de quoi la réunion va parler et quel est leur rôle.

\begin{itemize}
    \item Un courriel a été reçu.
    \item On peut demander l'ajout de sujets à aborder, en répondant.
    \item On doit prévenir si on ne vient pas, voir même si on vient. Sauf mentions contraires.
    \item On doit comprendre pourquoi on doit participer à la réunion. Si on ne comprend pas, il faut demander.
\end{itemize}

\subsubsection{Le rôle de participant}
Lorsqu'on est participant, on est porteur du projet ou rapporteur d'un rapport de projet. Il faut donc déterminer son positionnement et être cohérent avec son attitude.

\subsection{En tant qu'organisateur}
\begin{itemize}
    \item On doit préparer des convocations.
    \item On doit tout mettre sous écrit (courriel).
\end{itemize}

\subsubsection{La convocation}
Sur une convocation, on doit pouvoir retrouver les points suivants :
\begin{itemize}
    \item Le sujet (objectif) de la réunion.
    \item L'heure, la date, jour de la semaine et le lieu (de manière très précise).
    \item On doit montrer ou non la liste des participants.
    \item On ne doit pas envoyer la convocation ni trop tôt ni trop tard.
    \item On doit mettre \say{Si vous ne venez pas, merci de prévenir.}, pour éviter de recevoir trop de réponses.
    \item On doit mettre l'ordre du jour après le sujet (sous forme de liste numérotée).
\end{itemize}

\subsubsection{La réunion}
\begin{itemize}
    \item On doit préparer un scénario d'animation.
    \item Ainsi qu'une accroche, afin de mobiliser l'attention à la réunion.
    \item Vérifier les paramètres matériels, et choisir la bonne salle (lumière, chaises, tables, vidéo-projecteur, etc.).
    \item Si la réunion va dépasser les 40 minutes, mettre de l'eau à disposition.
    \item Éventuellement faire une pause si la réunion dure très longtemps (au moins avant 2h), par exemple avec des cafés, boissons, etc.
\end{itemize}

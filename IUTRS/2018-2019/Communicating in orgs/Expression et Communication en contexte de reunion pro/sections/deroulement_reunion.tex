\section{Une réunion réussie se déroule bien}
\subsection{Le début est la clé}
\begin{itemize}
    \item Il faut donner et fixer le cadre, afin d'éviter tout débordement sur le sujet.
    \item Il faut fixer l'ambiance (détendue, tendue, ...) en fonction du sujet.
    \item Il faut s'occuper de la répartition des rôles.
    \item Tout le monde doit voir de suite l'intérêt de la réunion. Il faut donc convaincre de suite qu'il y a un intérêt réel.
\end{itemize}

\subsection{Difficultés du début de réunion}
\subsubsection{Gestion des retards}
On ne peut éviter les retards, il y a toujours une très grande chance qu'une ou plusieurs personnes soient en retard. Pour éviter tout débordement, il faut suivre les quelques règles suivantes :

\begin{itemize}
    \item Si on veut attendre 5 minutes, il faut l'accord de tout le monde, si une personne ne peut attendre, il faut commencer. Mais, il faut bien entendu poser la question avec sous-entendues, afin d'éviter toute personne ayant peur de refuser.
    \item Si on est en retard, on doit :
    \begin{itemize}
        \item Toquer et entrer de suite (ne pas attendre).
        \item Dire bonjour et s'excuser.
        \item Ne pas donner la raison, cela interrompe la réunion (sauf attentat terroriste).
    \end{itemize}
    \item Il faut accueillir le retardataire.
    \begin{itemize}
        \item Accueillir (regard, geste, etc.) sans interrompt la réunion.
        \item Donner un court résumé dès que possible afin d'éviter que la personne soit perdue et qu'elle demande autour d'elle.
    \end{itemize}
\end{itemize}

\subsubsection{L'ordre du jour}
Au début de la réunion, il faut commencer par l'ordre du jour, et, donner l'accroche, afin que les gens n'entrent pas dans la pensée que la réunion est inutile.

Dans l'ordre du jour, il faut donner :
\begin{itemize}
    \item La durée,
    \item L'objectif,
    \item L'ordre du jour. \textbf{Mais}, ne pas donner trop de points sur une seule réunion (cinq points, maximum).
    \item Lister les points divers qui avaient été demandés.
\end{itemize}

Ensuite, il faut traiter l'ordre du jour avec un bon triage. Plutôt aller commencer par un point rapide puis un important. Et finir par un point rapide et léger.

\subsubsection{Présentation des participants ?}
\textbf{Seulement après l'accroche}, on peut présenter les participants rapidement si des personnes sont inconnues et nouvelles. Autrement, les personnes regarderont la ou les personnes en se demandant qu'est-ce cette personne a à faire ici.

\subsubsection{Désigner un secrétaire de séance}
Il faut une personne désignée afin qu'elle rédige la trace écrite de la réunion. S'il n'y a pas de personne désignée d'office, on demande un volontaire.

\subsection{Le scénario d'animation}
\subsubsection{Donner la parole aux gens à tour de rôle}
\begin{itemize}
    \item Il faut annoncer qu'on va essayer de faire court, dire qu'on va parler 10 minutes et laisser l'ouverture aux questions tout en laissant savoir qu'on a toujours pas mal de choses qu'on aurait pu dire.
    \item Éviter les \say{Je}, dire \say{Nous} à la place. Mais aussi les \say{Vous}.
    \item Solliciter les gens de manière personnelle, \say{Si vous avez des questions, n'hésitez pas}.
\end{itemize}

\subsubsection{Présentation avec support visuel}
Que cela soit un diaporama, un des papiers à faire passer, un tableau blanc ou un tableau à papier.
Les gens apprécient beaucoup les tableaux papier car ils donnent une très bonne interaction avec les gens et le matériel. Ce qu'un diaporama ne peut apporter.

\subsubsection{Brainstorming}
\begin{itemize}
    \item On parle quand on veut (tout le monde est actif).
    \item Mais il faut limiter le temps (moins de 15-20 minutes).
    \item Il faut au moins trois participants, autrement cela part facilement dans un débat.
\end{itemize}

\subsubsection{Subdivisons en petits groupes}
Une autre méthode est de subdiviser les gens en petits groupes puis faire un compte rendu à la fin. Cela permet d'éviter que chaque personne soit dans son coin et que l'organisateur doit gérer chaque personne.

\subsection{Une expression et communication efficaces}
\begin{itemize}
    \item Il faut faire attention à comment les messages sont réceptionnés (froncement de sourcils, regards, ...).
    \item Il faut regarder tout le monde, autrement on isole et on perd l'attention des gens.
    \item Il faut avoir une posture qui montre qu'on a envie d'être là.
    \item Il faut choisir les bons vêtements par rapport au sujet. Et ne jamais changer totalement de manière d'habillage, face à des gens qu'on voit tous les jours. Sinon, il faut prévenir les gens en avance qu'ils auront un choque durant la réunion.
    \item Il faut éviter les \say{Oui/non mais...}, qui transmettent l'effet de ne pas avoir été pris en compte.
    \item Il faut faire un petit geste ou, mieux, dire quelque chose pour récupérer la parole. On peut la reprendre avec un \say{Oui [...]} (attention à ne pas mettre de \say{mais} !).
    \item L'expression des objectifs doit être et ils doivent être dîts de façon S.M.A.R.T.
    \begin{description}
        \item[S] Personne spécifique, reconnaissance de la personne. Il faut faire en sorte que la réunion se passe bien pour tout le monde, et donc rendre tout le monde satisfait.
        \item[M] Quantifier l'objectif.
        \item[A] Objectif ambitieux.
        \item[R] Objectif réaliste.
        \item[T] Fixer la durée de l'objectif.
    \end{description}
\end{itemize}

\section{L'aspect physique de l'entretien}
\subsection{La préparation physique (aspect)}
On s'habille selon le type d'emploi. Pas besoin de venir en commercial, costard-cravate.

Il faut éviter les vêtements de sport, qui sont un loisir. Ne pas juste venir en t-shirt avec manches courtes, venir avec une veste ou quelque chose à manches longues au-dessus, pour éviter d'être associé au loisir.

\subsection{La présentation physique}
\subsubsection{La période d'attente}
Il faut arriver en avance, avoir le temps de se poser. Au moins dix minutes d'avance. Mais il sera très probablement en retard. On donne déjà une image de nous durant la période d'attente. Pas de casque sur les oreilles, pas d'écran, du moins pas de suite. Si l'attente est longue, on peut se familiariser avec le lieu, lire, etc.

\subsubsection{Durant l'entretien}
On n'est pas obligé de se mettre à la même place ou position que la personne précédente. On peut être déstabilisé par nos vêtements (vestes, écharpes, etc.), on peut la mettre sur le dossier de la chaise, mais jamais devant de soi. Puis en partant, les mettre sur le bras en veillant à ne rien faire tomber.

\subsubsection{La communication non verbale}
Toujours garder le contact. Essayer de parler aux trois ou quatre personnes devant nous, les regarder à tour de rôle. Montrer ses mains, elles sont associées au travail. Éviter de s'agiter.

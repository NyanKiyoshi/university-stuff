\section{Introduction}
On ne cherche pas seulement un emploi, on peut aussi passer des entretiens pour changer de poste (ex. devenir chef de service) mais donc, avec des gens qu'on connaît bien plus. Dans l'idéal on devrait en passer souvent, tous les deux trois ans.

\subsection{Avec qui}
\begin{itemize}
    \item Soit c'est \textbf{l'employeur} qui fait passer l'entretien,
    \item Soit c'est \textbf{le recruteur}, qui lui, est entraîné pour recruter un \say{commercial}.
\end{itemize}

Il faut donc savoir face à qui on va passer, car il ne faut pas dire la même chose, ni présenter de la même façon.

On est perçu comme un étranger total, on fait donc peur aux gens. Il faut donc inspirer la confiance.

\subsection{Types d'entretiens}
Il y a plusieurs types d'entretiens :
\begin{itemize}
    \item Les entretiens individuels (classiques) ;
    \item Les entretiens collectifs, où plusieurs candidats passent en même temps, mais va durer toute la journée voire plusieurs jours. On peut évaluer si les personnes savent travailler ensemble et sont efficaces, tout le monde est observé voire filmé, puis analysé.
    \item Avec ou sans exposé. Un entretien où on dit bonjour et on doit se présenter et donc se vendre. Si on le sait, on prépare un exposé. Si on le sait pas, vaut mieux en préparer un.
\end{itemize}

On peut demander discrètement si l'entretien sera un QA ou avec un temps de parole, et si c'est le cas, demander si on peut venir avec un support de présentation.

Il faudrait aussi demander s'ils ont déjà notre CV, ou s'ils ont en tête. Notamment pour pouvoir citer un point. Autrement, on sera dans l'idée qu'ils l'ont en tête alors que non.

\subsection{Pourquoi un entretien d'embauche}
Les compétences professionnelles ne sont pas les plus importantes, bien qu'elles répondent aux tâches, les relations humaines sont placées au-dessus. Il faut que la personne puisse apporter une certaines affection au sein de l'équipe de travail pour avoir de bonnes relations.

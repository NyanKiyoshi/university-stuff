\section{La préparation de l'entretien}
\subsection{Se renseigner sur l'entreprise ou le poste}

Il faut se renseigner sur l'entreprise pour montrer que cela nous intéresse, 

\begin{itemize}
    \item Trouver quelle est l'activité principale de l'entreprise, ne pas juste savoir vaguement.
    \item Leurs derniers gros projets, de ces dernières années. C'est leur monde, on leur donne l'impression de déjà se connaître.
\end{itemize}

Il y a aussi des informations secondaires, comme :
\begin{itemize}
    \item Le nombre d'employés pour éviter de se tromper. Par exemple dire par accident 100 employés à la place de 1000, et donc leur faire croire qu'on ne sait pas où on met nos pieds, leur entreprise n'est pas une PME.
    \item Le lieu, s'ils ont plusieurs places et s'ils ont des bureaux, où est-ce qu'on va travailler ?
    \item Combien de temps dure une journée de travail ?
    \item Quelle âge a l'entreprise ?
\end{itemize}

\subsubsection{Répondre aux questions}
Ne pas trop parler de soi, sinon cela renvoie une image égocentrique. Il faudrait plutôt répondre en parlant du poste, par exemple une entreprise en open-space, dire \say{J'aime bien travailler avec les autres}. Il ne faut pas s'éloigner du poste plus de 30 secondes, il ne faut surtout pas que cela devienne une conversation privée.

\subsection{Anticiper les questions}
Ne pas trop lire les réponses qui sont sur internet, elles sont clichés, ils le savent.

Comme dit précédemment c'est prévu pour des commerciaux. Et il faut encore une fois que le réflexe soit le poste, toujours le poste.

Il faut donner des défauts qui ne risquent pas d'affecter le travail (ex. gourmand, timide, ...). Dire qu'on est \say{trop} montre que c'est une qualité (ex. trop curieux, trop bavard).

\subsubsection{Les questions sur le CV}
Avoir le CV en tête, le réviser, pour éviter l'oublie. Éviter une réponse trop courte.

La question classique est \say{Parlez-nous de...}, par exemple \say{Parlez nous de cette année d'études}.

\subsection{La gestion de l'anxiété}
Une personne trop stressée ou nerveuse n'inspire pas la confiance.

\subsubsection{De quoi a-t-on peur ?}
\begin{itemize}
    \item De l'inconnu,
    \item Du jugement,
    \item De l'observation,
    \item Croire qu'on va être mal jugé, ce qui est faux.
\end{itemize}

\subsubsection{Ce qu'il faut faire}
Baisser la tension. On peut s'excuser au début de l'entretien qu'on est un peu nerveux et bafouillé, afin de prouver qu'on n'est pas comme ça dans la vie de tous les jours.

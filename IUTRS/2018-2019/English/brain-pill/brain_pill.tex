% !TeX spellcheck = en_US
% requires: texlive-latexextra texlive-pictures texlive-fontsextra
\documentclass[10.5pt]{article}

\usepackage{multicol}
\usepackage[a4paper, total={6.5in, 10in}]{geometry}
\usepackage[english]{babel}
\usepackage[utf8]{inputenc}
\usepackage[T1]{fontenc}
\usepackage[sfdefault]{roboto}
\usepackage{graphicx}
\usepackage{etoolbox}
\usepackage{qrcode}
\usepackage{vwcol}
\usepackage{float}
 
\makeatletter
  % change separation between dots
  \renewcommand\@dotsep{2.5} %default: 4.5
  
  % add dots to the section titles in the TOC
  \renewcommand*\l@section{\noindent\@dottedtocline{1}{1.5em}{2.3em}}
  \renewcommand*\l@subsection{\@dottedtocline{2}{3.8em}{3.2em}}
  \renewcommand*\l@subsubsection{\@dottedtocline{3}{7.0em}{4.1em}}
  \renewcommand*\l@paragraph{\@dottedtocline{4}{10em}{5em}}
  \renewcommand*\l@subparagraph{\@dottedtocline{5}{12em}{6em}}
\makeatother

\newcommand{\tm}{\emph{Brain Pill\nth{TM} }}
\newcommand{\softwareUrl}{https://brain-pill.vanille.bid/}

\newcommand{\nth}{\textsuperscript}
\makeatletter
\renewcommand\tableofcontents{%
  \null\hfill\textbf{\Huge Table Of Contents}\hfill\null\par\paragraph{}
  \@mkboth{\MakeUppercase\contentsname}{\MakeUppercase\contentsname}%
  \@starttoc{toc}%
}
\makeatother

\newenvironment{tightcenter}{%
  \setlength\topsep{0pt}
  \setlength\parskip{0pt}
  \begin{center}
}{%
  \end{center}
}
\newenvironment{bottompar}{\par\vspace*{\fill}}{\clearpage}

\title{Brain Pill(TM) - User Guide}
\author{KOCAK Mikail, GROSJEAN Martin}
\date{14\nth{th} September 2018}

\begin{document}
  % controlled by script over Wi-Fi/ bluetooth.
  % powered by....
  \vspace*{\fill}
  
  \centerline{\textbf{\centering\Huge B~R~A~I~N\ \ \ \ P~I~L~L  }}
  ~\\[0pt]
  \centerline{\noindent\rule{4in}{1pt}}
  ~\\[0pt]
  
  \noindent\centerline{\textbf{\centering\Huge User Guide}}
  ~\\[0pt]
  
  \vspace*{\fill}
  
  \begin{tightcenter}
    \large
    Please carefully read this guide before using our product\\
    and keep it for any future reference or issues.
  \end{tightcenter}
  
  \newpage
  
  % TOC
  \vspace*{\fill}
    \tableofcontents
  \vspace*{\fill}
  
  \newpage
  
  \section{Introduction}
  \tm is based on SSRIs to inject a micro-controller
  to one's hippocampus and take over the body remotely.
  
  \tm is endowed with a \emph{plug and play} technology allowing you
  to start controlling \tm within seconds
  \footnote{The pill requires two to three minutes to be fully active after swallowing.} by
  simply connecting a game controller to your device.
  
  \section{Quick Start}
    \begin{vwcol}[widths={0.85,0.15}, sep=.8cm, justify=flush,rule=0pt,indent=1em]
      \indent 1. Grab our software at \texttt{\softwareUrl} \\\indent\quad
                 or flash our \emph{QRCode};\\
      \indent 2. Swallow a pill;\\
      \indent 3. Wait 2 minutes;\\
      \indent 4. Plug in a game controller and enjoy!
      
      \vfill\eject
      \qrcode[height=.8in]{\softwareUrl}
    \end{vwcol}
    
    \begin{figure}[H]
    	\centering
      \includegraphics[width=.75\textwidth]{brainpill/ps3-brainpillcontroller.png}
      \caption[Close up of {Hemidactylus}]{controls for a PS4 game controller.}
    \end{figure}
  
  \section{Uses for \tm}
  \tm belongs to the \emph{Norwegian Psychiatrist Research Institute}, 
  which researched and developed this product for the sole objective 
  to allow people to overcome their fears and memories.\\
  
  \tm is internally and actively tested to treat:
  
  \begin{itemize}
    \setlength\itemsep{.25em}
    
    \item Acute Stress Disorders (ASR),
    \item Agoraphobia, 
    \item Claustrophobia,
    \item Obsessive-Compulsive Disorders (OCD),
    \item self-harm,
    \item pyromania, 
    \item social phobias,
    \item some forms of schizoaffective disorders.
  \end{itemize}
  
  \section{Before Using \tm}
  Before anything, the risks must be weighed against the good the pill will do to one's body.
  Taking this pill is a decision only for you and your doctor to make.
  
  To lower the risk of memory loss, 
  you only should take this pill when you have had a full night of sleep (at least 6 or 7 hours).
  
  Always put at least a 5 hour gap between each \tm usage.
  
  \subsection{Allergies}
    Tell your doctor if you encounter any unusual allergic reaction during 
    or after the pill' effects.
    
  \subsection{Pediatric and Breast Feeding}
    Studies on \tm were only done on adult psychiatric patients. 
    Thus, no information or studies are available on the usage risks for children or infants.
    
  \subsection{Storage}
    The pills must be kept in a closed container at room temperature 
    away from children, liquids, moisture, freezing and direct light.
    
  \section{Precautions While Using \tm}
    Make sure you are in a safe environment, where you cannot easily be hurt.
    We also recommend that you stay seated on a chair or in bed 
    when waiting for the pill to become active.
    
    \tm may cause some people to feel drowsy, dizzy, lightheaded, clumsy or unsteady.
    Make sure, when getting up, to carefully take your time 
    and do not hesitate to abort if you are feeling heavy or at risk.

    As the pill is flagged as an unknown entity, thus, as a normal reaction, 
    will be slowly eliminated. The the effects will slowly withdraw. 
    From our experiments, the pills are lasting at least an hour and up to two hours.
    
  \section{\tm Side Effects}
    Along with its desired effects, \tm can come with some unwanted effects.
    If anything uncommon occurs, make sure to check with your doctor as soon as possible.\\
    
    \noindent Most common:
    \begin{itemize}
      \item Confusion;
      \item Clumsiness or unsteadiness;
      \item Difficulty with coordination;
      \item Mood or mental changes.
    \end{itemize}
    
    \noindent Less common:
    \begin{itemize}
      \item Severe drowsiness;
      \item Shortness of breath and difficult or labored breathing;
      \item Tightness in the chess;
      \item Memory loss.
    \end{itemize}
    
    \noindent Rare:
    \begin{itemize}
      \item Loss of consciousness;
      \item Increased appetite;
      \item Weight loss;
      \item Vomiting.
    \end{itemize}
  
\end{document}

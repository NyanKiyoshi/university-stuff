\subsection{Théorème fondamental de l'arithmétique}

\subsubsection{Définition}
    Un entier positif est dit \textbf{premier} s'il possède exactement deux
    diviseurs positifs.
        
    Les résultats suivants sont démontrés dans les \emph{Éléments} d'Euclide,
    IV\textsuperscript{e} s.
    
\subsubsection{Proposition}
    \begin{enumerate}
        \item Soit $a$ un entier positif :
             \begin{itemize}
                 \item ou bien $a$ est premier,
                 \item ou bien $a$ admet un diviseur premier inférieur ou égal à
                    $\sqrt{a}$.
             \end{itemize}
         \item Il existe une infinité de nombres premiers.
    \end{enumerate}
    
\subsubsection{Démonstration}
    \begin{itemize}
        \item En effet, si $a$ n'est pas premier, il admet des diviseurs,
            et le plus petit d'entre eux sera premier...
        \item Par l'absurde, en considérant le produit de tous les premiers
            augmenté de $1$...
    \end{itemize}
    
\subsubsection{Théorème (Th. Fondamental de l'arithmétique)}
    Soit $\mathcal{P}$ l'ensemble (infini) des nombres premiers,
    tout entier positif admet une décomposition unique, à l'ordre près
    des facteurs, comme produit de nombre premiers :
    
    \[
        n = \prod_{p \in \mathcal{P}}^{} p^{v_p(n)},
    \]
    
    où les entiers $v_p(n)$ sont nuls sauf pour un nombre fini de premiers $p$.
    
    \noindent\textbf{Démonstration}\\
    \indent L'existence est une récurrence facile sur les entiers, l'unicité est la partie
    la plus « forte » de l'énoncé : c'est aussi une récurrence sur les entiers,
    où le cas de « base » est celui des nombres premiers, et où l'étape de
    récurrence s'appuie sur le lemme de Gauß.
    
    \noindent\textbf{Remarque}\\
    \indent Il n'y a pas d'algorithme efficace connu permettant de générer les nombres
    premiers, ou de factoriser un entier.
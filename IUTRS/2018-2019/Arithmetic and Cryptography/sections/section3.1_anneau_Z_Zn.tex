% https://web.archive.org/web/20180429012925/http://mathonline.wikidot.com/the-ring-of-z-nz
% https://people.math.umass.edu/~hajir/m499c/m499c-hw1.pdf
\subsection{L'anneau $\Z/n\Z$ -- Les entiers modulaires $n$}

\subsubsection{Définition}
    Soit $n \in \N^*$, on définit la relation suivante sur $\Z$ :
    \[
        x \equiv y \mod n \iff n | (x-a)
    \]
    
    L'ensemble des classes d'équivalences est noté $\Z/n\Z$. On note, 
    $ \dot{a} \in \Z/n\Z $, ou encore $ \bar{a} \in \Z/n\Z $, 
    la classe de l'entier $ a \in \Z $.

\subsubsection{Proposition}
    Pour tous entiers $a$, et $b$, on a une relation d'équivalence de :
    \begin{enumerate}
        \item $a \equiv b \mod n$
        \item $a$ et $b$ ont le même reste dans la division euclidienne par $n$.
    \end{enumerate}

\subsubsection{Caractéristiques}
    \begin{description}
        \item[Réflexivité :] $n | 0 = a - a$, donc $a \equiv a[n]$ ;
        \item[Surjection :]
            \begin{flalign*}
                a \equiv b[n] &\iff n / (b-a) && \\
                              &\iff n / (a-b) && \\
                              &\iff b \equiv a[n] &&
            \end{flalign*}
    \end{description}
    
    Par conséquent, un système "naturel" de représentants des classes
    est l'ensemble des entiers compris entre $0$ et $n-1$ :
    chaque entier est représenté par son reste dans la division euclidienne 
    par $n$.\\
    
    \noindent\textbf{Remarque :}
        \noindent Il peut être intéressant (surtout pour les calculs "à la main"),
        de travailler avec un autre système de représentants :
        
        \begin{align*}
            \Z/5\Z &= \{\bar{0}, \bar{1}, \bar{2}, \bar{3}, \bar{4}\} \\
                   &= \{\bar{-2}, \bar{-1}, \bar{0}, \bar{1}, \bar{2}\}
        \end{align*}

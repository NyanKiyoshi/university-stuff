\subsection{Divisibilité}

\subsubsection{Le cadre}
    \begin{itemize}
        \item $\N$, des entiers naturels, est un ensemble
            \textbf{bien ordonné} ;
        \item $\Z$, des entiers relatifs, forme un
            \textbf{anneau commutatif}. Ce qui impliques :
            \begin{itemize}
                \item Les lois associatives ;
                \item Les lois commutatives.
            \end{itemize}
    \end{itemize}
    
    ~\\ \noindent\textbf{Exemple de structure d'anneau}\\
    \indent $ a \times (b + c) = a \times b + a \times c $,
    c'est une structure d'anneau, car un $+$ est lié à un $\times$.
    
\subsubsection{Définition : Relation de divisibilité}
    Soit $a$ et $b$, deux entiers. On dit que \textbf{$a$ divise $b$},
    ou que $a$ est un diviseur de $b$, ou encore que $b$ est un multiple
    de $a$. Et on écrit $a | b$, s'il existe un entier $k$ tel que $b = ka$.\\
    
    \noindent\textbf{Remarques}
    \begin{itemize}
        \item Zéro n'est diviseur d'aucun entier, et multiple de tous ;
        \item Tous les entiers sont diviseurs de zéro ;
        \item Un entier $a$ est toujours divisible par $1$ et par $a$
        (ainsi que par $-1$ et $-a$).
    \end{itemize}
    
\subsubsection{Les nombres premiers}
    \noindent\textbf{Définition}
    \begin{enumerate}
        \item Soit $p$, un entier positif, on dit que $p$ est premier 
            \textbf{si ses seuls diviseurs positifs sont $1$ et lui-même} ;
        \item Soit $a$ et $b$, deux entiers, on dit que $a$ et $b$
            sont premiers entre eux \textbf{quand leur seul diviseur commun positif est $1$}.
    \end{enumerate}
    
\subsubsection{Plus grand commun diviseur (PGCD)}
    Soit $a$ et $b$, deux entiers. L'ensemble des diviseurs positifs communs
    à $a$ et $b$ n'est vide (il contient au moins $1$), et fini
    (tout diviseur de $a$ est inférieur à $|a|$).
    
    Il admet donc un plus grand élément : le plus grand diviseur commun
    à $a$ et $b$, noté $pgcd(a, b)$.

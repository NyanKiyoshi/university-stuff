\section{Introduction sur la crypto}

\subsection{De l'antiquité au XIX\textsuperscript{e} siècle}
    \begin{itemize}
        \item Dans un objectif de confidentialité ;
        \item Le chiffrement repose sur des permutations et des
            substitutions (mono ou poly) alphabétiques ;
        \item C'est un artisanat, et non pas un art scientifique
            et c'est un secret.
    \end{itemize}

\subsection{L'ère de la guerre industrielle}
    \begin{itemize}
        \item Les communications sont importantes ;
        \item La cryptographie progresse ;
        \item Le premier texte énonçant des principes systématiques,
            il est connu sous le nom du \emph{Principe de Kerckhoffs},
            qui en résumé, dit :
            
            \quad \boxed{
                \texttt{Le secret doit résider dans la clef et non dans le procédé
                de chiffrement.}
            }
            \cite{Auguste Kerckhoffs}
            
            C'est donc plus facile à changer. Par exemple, communiquer
            à 100000 hommes une clef plutôt qu'une nouvelle méthode complexe.
        \item \textbf{Confusion, Diffusion,} 
            changer au maximum le message dès lorsqu'un bit change.
    \end{itemize}
    
\subsection{Les années 70 : double révolution}
    \begin{itemize}
        \item Explosion des besoins en cryptographie à clé publique ;
        \item Premier standard de chiffrement par blocs plus ou moins
            universel : DES\footnote{Data Encryption Standard} (RSA),
            qui simplifie les communications, tout le monde se met d'accord.
    \end{itemize}
    
\subsection{Cryptographie à clef publique}
    \begin{itemize}
        \item Facile à calculer ;
        \item Difficile à dé-calculer, \textbf{sauf si on sait une chose en plus.}
    \end{itemize}
    
\subsection{Standard de chiffrement par blocs}
    \begin{itemize}
        \item 1977 DES est né ;
        \item 1990, DES est déclaré trop vulnérable au brute force,
            car la clef est trop courte (56 bits) ;
        \item 2000, un concours international est lancé pour choisir un nouvel
            algorithme. L'algorithme de Rijndael (aujourd'hui nommé AES) est gagnant.
        \item Les clés AES sont de 128, 192 ou 256 bits. Ce qui rend les attaques
            brute force quasiment physiquement impossibles.
    \end{itemize}
    
\subsection{Aujourd'hui}
    \begin{itemize}
        \item On veut de la confidentialité ;
        \item On veut assurer l'intégrité des échanges 
            (ne pas pouvoir modifier le message) ;
        \item L'authentification (y compris signature),
            on doit assurer que le destinataire est le vrai.
    \end{itemize}
    
    ~
    
    Le certificat de la clef publique permet d'assurer que seule cette
    personne saura déchiffrer le message. Le certificat est vérifié via
    la signature.
    
\vspace*{\fill}
\begin{thebibliography}{1}
    \bibitem{Auguste Kerckhoffs}
    Auguste Kerckhoffs
\end{thebibliography}

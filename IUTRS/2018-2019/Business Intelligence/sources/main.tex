% !TeX spellcheck = fr_FR
\documentclass[final, a4paper, 11pt]{article}
\usepackage[french]{babel}
\usepackage[utf8]{inputenc}
\usepackage[T1]{fontenc}
\usepackage{fontspec}
\usepackage{fullpage}
\usepackage{float}
\usepackage{mdframed}
\usepackage[margin=2cm]{geometry}
\usepackage{amsmath}
\usepackage{dirtytalk}
\usepackage{listings}

\definecolor{dkgreen}{rgb}{0,0.6,0}
\definecolor{gray}{rgb}{0.5,0.5,0.5}
\definecolor{mauve}{rgb}{0.58,0,0.82}
\lstset{
    aboveskip=3mm,
    belowskip=3mm,
    showlines=true,
    showstringspaces=false,
    emptylines=2,
    columns=flexible,
    basicstyle={\small\ttfamily},
    numbers=none,
    numberstyle=\small, 
    numbersep=8pt,
    numbers=left,
    frame=single,
    numberstyle=\tiny\color{gray},
    keywordstyle=\color{blue},
    commentstyle=\color{dkgreen},
    stringstyle=\color{mauve},
    breaklines=true,
    breakatwhitespace=true,
    tabsize=4,
    language=sql
}


%%% Commands
\newcommand{\code}[1]{
    \boxed{\texttt{#1}}
}


\begin{document}

\noindent
\large\textbf{C41 - Expression et Communication en ctx de réunion Pro} \hfill \textbf{KOCAK Mikail} \\
\normalsize M. Eric Wessler-Laux \hfill 1 février 2019 \\[1pt]


\begin{description}
    \item[Pourquoi ?]
    C'est un besoin, gain de temps.
    
    \item[Déroulement du module]
    Deux premières semaines : compréhension et fouille de données.
    Dernière séance : contrôle des connaissances du cours et du mini projet.
\end{description}


\section{Qu'est-ce que la fouille de données ?}
On veut sortir quelque chose de nos données via des statistiques. Apprentissage automatique via intelligence artificielle (deep learning), beaucoup de données et beaucoup de calculs.

Aucune méthode d'apprentissage ne peut être la meilleure. Chacune a ses avantages.

Durant les années 90, des logiciels ont été développés pour tirer des connaissances des bases de données. La puissance de calcul des machines qui était déjà très bonne.

Data mining, les données sont des matériaux brutes, c'est de la terre, de la roche, il faut extraire, ça ne pousse pas tout seul.

Mais on découvre que ces données ne sont pas terribles, mais ce qui nous intéresse c'est de faire des formes de prédiction via l'apprentissage de règles qui conduit à un modèle qui s'applique à tout les clients. Le deep learning peut donc être de trop.

Le problème est qu'il faut préparer les données. Car elles ne sont pas forcément conçus pour la fouille de données. Mais il faut aussi nettoyer la base de données, qui est polluée de bêtises et de fausse données (le stagiaire doit s'occuper de ça). C'est 80\% du temps de travail. Les 20\% restants vérifient que le modèle généré est correcte. La génération du modèle est quasi instantané.

On a rarement trop de données.

\say{Data scientist}, c'est la science des données.

\subsection{Découvrir des motifs}
Un exemple est \say{l'analyse du panier de la ménagère}, on analyse le cadis des gens, via le ticket de caisse des produits achetés en même temps. On regarde les données et on les généralises.

L'objectif du patron du super marché est d'augmenter le pouvoir d'achat. Une solution est d'augmenter le salaires de clients, mais le patron du super marché ne peut pas. Il doit donc trouver d'autres solutions.

Une chaîne de magasin Américaine a découvert que lorsqu'une personne achète des couches pour bébé, elle achète des bières. C'est quelque chose de nouveau, qu'on n'avait connaissance. C'est à la chaîne de prendre une initiative, de quoi faire de cette information.

Bien entendu, on se fiche de savoir qui a acheté les couches pour bébé. On veut juste tirer des règles.

\subsection{Définition de l'apprentissage automatique}
On dit qu'un programme informatique apprend, à partir d'une expérience $E$, par rapport à une classe de tâches $T$ et une mesure de performance $P$, si sa performance sur des tâches de $T$, mesurée par $P$, s'améliore avec l'expérience $E$.

\subsection{Exemple introductif}
On veut savoir quand est-ce qu'il joue à partir des données ci-dessous, donc, en tirant des règles.
\begin{table}[H]
    \centering
    \begin{tabular}{|l|l|l|l|l|l|}
        \hline
        Jour & Ciel & Température & Humidité & Vent & Jouer \\ \hline
        J1  & Soleil & Chaud & Élevée & Faible & Non  \\ \hline
        J2  & Soleil & Chaud & Élevée & Fort & Non \\ \hline 
        J3  & Couvert & Chaud & Élevée & Faible & Oui \\ \hline
        ... & ... & ... & ... & ... & ... \\ \hline
    \end{tabular}
    \caption{Exemple de données à fouiller}
    \label{tab:my_label}
\end{table}

\subsection{Exemples d'applications réelles}
\begin{itemize}
    \item Aide à la décision (attribution de prêts bancaires, cartes de crédit, etc.)
    \item Analyse d'images (détection de nappes de pétrole, structures astronomiques, médical, etc.)
    \item Apprendre à reconnaître des mots parlés
    \item Apprendre à conduire un véhicule autonome
    \item Apprendre à jouer au backgammon à un niveau de champion
    \item Commerce et marketing (panier de la ménagère, ticket de caisse, carte de fidélité, mailing, etc.)
\end{itemize}

La carte de fidélité est pratique car à la place d'avoir un seul ticket, on a un ensemble de ticket et on incite le client à revenir. On se fiche du nom, on veut juste pouvoir associer les tickets à une carte.

On veut faire plaisir au client, c'est notre rôle.

On collabore au big brother.

\section{Différents types d'apprentissages}

\subsection{Apprentissage supervisé (ce n'est pas interactif)}
On a une classe, un attribut de sortie. Il y a une entrée, une sortie.

\begin{itemize}
    \item La majorité des exemples précédents ;
    \item On a une classe ;
    \item Exemples étiquetés par un professeur (par un oracle) ;
    \item Classe numérique (régression) ou catégorielle (classification).
\end{itemize}

\subsection{Apprentissage non-supervisé}
C'est le cas où on n'a pas de classe prédéfinie.

Deux méthodes :
\begin{description}
    \item[Découverte de règles] on veut découvrir des règles, peu importe leur conclusion.
    \item[Clustering,] on veut pouvoir trier les données par paquet de similarité (qu'on a définie). Permet donc une représentation des données. On crée donc un ensemble de classes à analyser.
\end{description}

\section{La fouille de données}
\subsection{Apprentissage de concepts}
\begin{description}
    \item[Fonction booléenne,] Par exemple il pleut, il pleut pas.
    \item[Représentation des hypothèses : ] une règle. Si conjonction d'attributs-valeurs, alors appartient au concept. 
    \item[Espace des hypothèses,] combien de règles on peut avoir. $ 4 \times 4 \times 3 \times 3 = 144 \text{~conjonctions} $.
    \item[Espace des instances,] combien de représentations possibles. $ 3 \times 3 \times 2 \times 2 = 36 \text{~instances} $.
\end{description}

\subsection{Le principe de l'apprentissage inductif}
Toute hypothèse qui est une bonne approximation de la fonction cible sur l'ensemble d'apprentissage est aussi une bonne approximation de la fonction cible sur les instances non observées.

\subsection{Parcours de l'espace des hypothèses}
\begin{itemize}
    \item On énumère et on garde ce qui est positif.
    \item Relation d'ordre.
    \item Recherche de toutes les hypothèses cohérentes. 
\end{itemize}

\subsection{Problèmes}
Il reste plusieurs hypothèses cohérentes avec l'ensemble d'apprentissage. Il y a pas assez d'exemple pour isoler la "bonne" hypothèse. Critère supplémentaire pour choisir la "bonne".

Il n'y a plus d'hypothèse cohérente avec l'ensemble d'apprentissage. Soit le langage n'est pas assez expressif, soit les exemples sont bruités (ex.: un jour où il joue pas malgré qu'il fait moche).

\subsection{Le bruit en apprentissage}
\begin{itemize}
    \item Origines diverses (physiques, expérimentales, humaines, etc.) ;
    \item Mauvaise classe associée à la description.
\end{itemize}

\subsection{Pas d'apprentissage sans biais}
\begin{itemize}
    \item $ 2^{36} = 6.9 \times 10^{10} $ (69MM) concepts possibles.
    \item En fait, $ 2^{36 - 14} = 4096 $ concepts cohérents avec les exemples.
    \item Il est impossible de choisir sans biais supplémentaire.
    \item Biais de langage.
    \item Biais de recherche.
    \item Biais pour éviter le sur-apprentissage.
\end{itemize}

\subsection{La méthodologie de la fouille de données}
\begin{enumerate}
    \item Identifier le problème à résoudre.
    \item Préparer les données d'entrée.
    \item Explorer des modèles multiples.
    \item Utiliser le modèle sur le réel.
    \item Suivre le modèle et l'améliorer.
\end{enumerate}

\subsection{Identifier le problème à résoudre}
\subsubsection{Concept}
Reformuler le problème en un des types connus.

\subsubsection{Entrées}
\begin{itemize}
    \item Attributs........
\end{itemize}

\newpage
\section{OLAP}

\subsection{Fonction d'agrégation}

\subsubsection{Appel par ligne}
\begin{lstlisting}[title=Afficher le numéro de ligne de chaque entrée.]
     SELECT ROW_NUMBER() OVER() AS row_num, order_no, price FROM aroma.orders;
\end{lstlisting}

\begin{lstlisting}[language=]
     ROW_NUM              ORDER_NO    PRICE    
-------------------- ----------- ---------
                   1        3602    780.00
                   2        3604    800.66
                   3        3615   4600.10
                   4        3623   4425.00
                   5        3624   4425.00
                   6        3625   3995.95
                   7        3621  10234.50
\end{lstlisting}

\subsubsection{Appel sur l'ensemble des lignes précédentes}
\code{OVER(...)} nous permet d'exécuter une fonction OLAP ou du SGBD sur chaque ligne, plutôt que sur l'ensemble des lignes.

\begin{lstlisting}[title=Faire la somme des moyennes au-dessus de la ligne.]
SELECT SUM(AVG(dollars)) OVER(ORDER BY dollars ROWS UNBOUNDED PRECEDING)
FROM aroma.sales GROUP BY dollars ORDER BY dollars;
\end{lstlisting}

\begin{lstlisting}[language=]
     1.75000000000000000000000000
     3.75000000000000000000000000
     6.00000000000000000000000000
     8.50000000000000000000000000
    11.25000000000000000000000000
    14.25000000000000000000000000
    17.50000000000000000000000000
    21.00000000000000000000000000
\end{lstlisting}

\begin{itemize}
    \item \code{OVER(...)} indique au SGBD que la fonction \code{SUM} est une fonction d'agrégation OLAP.
    \item \code{ROWS UNBOUNDED PRECEDING} permet d'instruire OLAP de prendre les lignes du dessus.
    \item Le \code{ORDER BY} d'OLAP (le premier, donc) est obligatoire, car il assure que les lignes d'entrée sont correctement triées. Autrement, les résultats peuvent prendre aucun sens logique.
    \item Le dernier \code{ORDER BY} permet de modifier l'affichage des résultats, et donc de distinguer les résultats depuis le \code{ORDER BY} de OLAP .
\end{itemize}

\subsubsection{Appel sur l'ensemble des lignes précédentes par groupe donné}
\begin{lstlisting}
SELECT date, SUM(dollars) FROM aroma.period a, aroma.sales b, aroma.product c 
WHERE a.perkey = b.perkey AND c.prodkey = b.prodkey AND c.classkey = b.classkey 
GROUP BY week, date ORDER BY week, date;
\end{lstlisting}

Permet de faire la somme des ventes sur une journée.

\begin{lstlisting}
 SELECT date, SUM(dollars), SUM(SUM(dollars)) OVER(PARTITION BY week ORDER BY date ROWS UNBOUNDED PRECEDING) FROM aroma.period a, aroma.sales b, aroma.product c WHERE a.perkey = b.perkey AND c.prodkey = b.prodkey AND c.classkey = b.classkey GROUP BY week, date ORDER BY week, date;

\end{lstlisting}

\begin{lstlisting}[language=]
DATE       2                                 3                                
---------- --------------------------------- ---------------------------------
01/02/2004                           8851.70                           8851.70
01/03/2004                           8039.00                          16890.70
01/01/2005                           8988.60                          25879.30
01/01/2006                          11529.50                          37408.80
01/02/2006                           8795.00                          46203.80
01/03/2006                           7825.20                          54029.00
01/04/2006                           8849.50                          62878.50
01/05/2006                           9964.10                          72842.60
01/06/2006                           7788.50                          80631.10
01/07/2006                           9598.90                          90230.00
01/04/2004                           7828.35                           7828.35
01/05/2004                           9036.95                          16865.30
01/06/2004                          10089.70                          26955.00
01/07/2004                           6798.75                          33753.75
\end{lstlisting}

\code{PARTITION BY} nous permet de faire la somme des lignes précédentes et se remet à zéro dès que la semaine change.

\subsection{Opérations arithmétiques}
\subsubsection{Opérations}
Les opérations standards sont supportées (*, /, +, -, (, )).

\begin{lstlisting}[title=Prix moyen de chaque produit par vente]
SELECT date, SUM(dollars) as TotalSales, SUM(quantity) AS TotalQ,
    SUM(dollars) / SUM(quantity)
FROM aroma.period a, aroma.sales b, aroma.product c 
WHERE a.perkey = b.perkey AND c.prodkey = b.prodkey AND c.classkey = b.classkey 
GROUP BY week, date ORDER BY week, date;
\end{lstlisting}

\begin{lstlisting}[language=]
DATE       TOTALSALES                        TOTALQ      4                                
---------- --------------------------------- ----------- ---------------------------------
01/02/2004                           8851.70        1246                              7.10
01/03/2004                           8039.00        1212                              6.63
01/04/2004                           7828.35        1067                              7.33
01/05/2004                           9036.95        1261                              7.16
01/06/2004                          10089.70        1390                              7.25
01/07/2004                           6798.75        1014                              6.70
01/08/2004                           7989.05        1024                              7.80
01/09/2004                           8416.45        1179                              7.13
01/10/2004                           5809.80         841                              6.90
01/11/2004                           7483.55        1109                              6.74
01/12/2004                           9423.05        1272                              7.40
01/13/2004                           7890.85        1117                              7.06
01/14/2004                           8324.95        1142                              7.28
01/15/2004                           7693.70        1089                              7.06
01/16/2004                           8270.40        1117                              7.40
01/17/2004                           6564.00         928                              7.07
01/18/2004                           9736.95        1328                              7.33
\end{lstlisting}

\subsubsection{Changement de précision décimale}
\code{DEC(VAL, NB\_DIGITS\_BEFORE, NB\_AFTER)}, par exemple \code{DEC(1, 5, 2)}.

% Section 5 is going on here
\subsection{Comparaison de deux totaux}
Pour cela, on va pouvoir faire deux jointures (sur les dates) et un soustraction sur les résultats.

\begin{lstlisting}
SELECT t1.date, sales_cume_west, sales_cume_south,
                sales_cume_west - sales_cume_south AS west_vs_south
FROM
    (SELECT date, SUM(dollars) AS total_sales,
        SUM(SUM(dollars)) OVER(ORDER BY date
            ROWS UNBOUNDED PRECEDING) as sales_cume_west
        FROM aroma.market a,
             aroma.store b,
             aroma.sales c,
             aroma.period d
        WHERE  a.mktkey = b.mktkey
            AND b.storekey = c.storekey
            AND d.perkey = c.perkey
            AND year = 2006
            AND month = 'MAR'
            AND region = 'West'
        GROUP BY date) AS t1
JOIN
    (SELECT date, SUM(dollars) AS total_sales,
        SUM(SUM(dollars)) OVER(ORDER BY date
            ROWS UNBOUNDED PRECEDING) AS sales_cume_south
        FROM aroma.market a,
             aroma.store b,
             aroma.sales c,
             aroma.period d
        WHERE a.mktkey = b.mktkey
            AND b.storekey = c.storekey
            AND d.perkey = c.perkey
            AND year = 2006
            AND month = 'MAR'
            AND region = 'South'
        GROUP BY date) AS t2
ON t1.date = t2.date
ORDER BY date;
\end{lstlisting}

\begin{lstlisting}[language=]
DATE       SALES_CUME_WEST           SALES_CUME_SOUTH      WEST_VS_SOUTH         
---------- ------------------------- -------------------- --------------
03/01/2006       2529.25              2056.75                472.50
03/02/2006       6809.00              4146.75               2662.25
03/03/2006       9068.75              6366.55               2702.20
03/04/2006      12679.35              8831.30               3848.05
03/05/2006      16228.60             11100.55               5128.05
03/06/2006      19653.30             12665.65               6987.65
03/07/2006      23515.55             14882.90               8632.65
03/08/2006      27207.80             17494.15               9713.65
03/09/2006      31725.95             19745.40              11980.55
03/10/2006      34836.20             21323.40              13512.80
03/11/2006      38257.00             23256.15              15000.85
03/12/2006      42498.20             25091.35              17406.85
03/13/2006      46691.95             27362.85              19329.10
\end{lstlisting}

\subsection{La moyenne des sommes par partition}
Ici on veut calculer la moyenne des ventes par ville par trois semaines. Pour cela, on fera :

\begin{lstlisting}
SELECT city, week, SUM(dollars) AS sales,
    AVG(SUM(dollars)) OVER(PARTITION by city 
        ORDER BY city, week ROWS 2 PRECEDING) AS mov_avg,
    SUM(SUM(dollars)) OVER(PARTITION BY city
        ORDER BY week ROWS unbounded PRECEDING) AS run_sales
FROM aroma.store a,
     aroma.sales b,
     aroma.period c
WHERE a.storekey = b.storekey
    AND c.perkey = b.perkey
    AND qtr = 'Q3_05'
    AND city IN ('San Jose', 'Miami')
GROUP BY city, week;
\end{lstlisting}

\begin{lstlisting}[language=]
CITY           WEEK        SALES       MOV_AVG      RUN_SALES                        
------------ ----------- ---------     ----------  ----------------
Miami          27         1838.55       1838.55       1838.55
Miami          28         4482.15       3160.35       6320.70
Miami          29         4616.70       3645.80       10937.40
Miami          30         4570.35       4556.40       15507.75
Miami          31         4681.95       4623.00       20189.70
Miami          32         3004.50       4085.60       23194.20
Miami          33         3915.90       3867.45       27110.10
Miami          34         4119.35       3679.91       31229.45
Miami          35         2558.90       3531.38       33788.35
Miami          36         4556.25       3744.83       38344.60
Miami          37         5648.50       4254.55       43993.10
Miami          38         5500.25       5235.00       49493.35
Miami          39         4891.40       5346.71       54384.75
Miami          40         3693.80       4695.15       58078.55
San Jose       27         3177.55       3177.55       3177.55
San Jose       28         5825.80       4501.67       9003.35
San Jose       29         8474.80       5826.05       17478.15
San Jose       30         7976.60       7425.73       25454.75
San Jose       31         7328.65       7926.68       32783.40
San Jose       32         6809.75       7371.66       39593.15
San Jose       33         7116.35       7084.91       46709.50
San Jose       34         6512.35       6812.81       53221.85
San Jose       35         6911.50       6846.73       60133.35
San Jose       36         5996.10       6473.31       66129.45
San Jose       37         10000.60      7636.06       76130.05
San Jose       38         7274.70       7757.13       83404.75
San Jose       39         9021.15       8765.48       92425.90
San Jose       40         5045.20       7113.68       97471.10
\end{lstlisting}

Noter le \code{ROWS n PRECEDING} qui indique qu'on veut regarder les \code{n} lignes précédentes + la ligne courante.

\subsection{Classer les données par rang}
On va classer les magasins qui vendent les plus.

\begin{lstlisting}
SELECT store_name, district, SUM(dollars) AS total_sales,
    RANK() OVER(ORDER BY SUM(dollars) DESC) AS rank
FROM aroma.market a,
     aroma.store b,
     aroma.sales c,
     aroma.period d
WHERE a.mktkey = b.mktkey
    AND b.storekey = c.storekey
    AND d.perkey = c.perkey
    AND year = 2005
    AND month = 'MAR'
    AND region = 'West'
GROUP BY store_name, district;
\end{lstlisting}

\begin{lstlisting}[language=]
STORE_NAME                     DISTRICT             TOTAL_SALES                       RANK                
------------------------------ -------------------- ---------------------------------
Cupertino Coffee Supply        San Francisco                                 18670.50                    1
Java Judy's                    Los Angeles                                   18015.50                    2
Beaches Brew                   Los Angeles                                   18011.55                    3
San Jose Roasting Company      San Francisco                                 17973.90                    4
Instant Coffee                 San Francisco                                 15264.50                    5
Roasters, Los Gatos            San Francisco                                 12836.50                    6
\end{lstlisting}

Noter, qu'il existe \code{DENSE\_RANK} qui fait la même chose que \code{RANK}, sauf qui si plusieurs lignes ont le même rang, le rang suivant sera incrémenté par rapport au nombre de rang identiques.

\begin{lstlisting}[language=]
DENSE_RANK     RANK
1              1
2              2
3              3
3              3
5              4
6              5
\end{lstlisting}

\subsection{Opérations sur les dates}
Avec les fonctions OLAP, on peut facilement faire des calculs arithmétiques sur des dates.

\begin{lstlisting}
SELECT date - 2 MONTHS - 10 DAYS as due_date,
    date AS cur_date,
    date + 2 MONTHS + 10 DAYS AS past_due
FROM aroma.period
WHERE year = 2004
    AND MONTH = 'JAN';
\end{lstlisting}

\begin{lstlisting}[language=]
DUE_DATE   CUR_DATE   PAST_DUE  
---------- ---------- ----------
10/22/2003 01/01/2004 03/11/2004
10/23/2003 01/02/2004 03/12/2004
10/24/2003 01/03/2004 03/13/2004
10/25/2003 01/04/2004 03/14/2004
10/26/2003 01/05/2004 03/15/2004
\end{lstlisting}

Idem pour \code{YEARS}, \code{MONTHS}, \code{DAYS}, \code{HOURS}, \code{MINUTES}, et \code{SECONDS}.

\subsection{Having et Where}
\begin{description}
    \item[Having] agit sur les données après le groupage et on peut utiliser toute fonction, colonne ou ensemble de données.
    
    \item[Where] agit sur les données avant le groupage, mais ne peut pas utiliser les fonctions d'ensembles (SUM, AVG, ...).
\end{description}

\subsubsection{Exemple}
\begin{lstlisting}
SELECT MIN(prodkey), MAX(classkey)
FROM aroma.product
HAVING MIN(prodkey) = 0;
\end{lstlisting}

\begin{lstlisting}[language=]
1           2          
----------- -----------
          0          12
\end{lstlisting}

\subsection{Faire la somme par valeur de champ via un switch-case}
\begin{lstlisting}
SELECT prod_name,
    SUM(CASE WHEN store_name = 'Beaches Brew'
        THEN dollars ELSE 0 END) AS Beaches,
    SUM(CASE WHEN store_name = 'Cupertino Coffee Supply'
        THEN dollars ELSE 0 END) as Cupertino
FROM aroma.market a,
     aroma.store b,
     aroma.period c,
     aroma.product d,
     aroma.class e,
     aroma.sales f
WHERE a.mktkey = b.mktkey
    AND b.storekey = f.storekey
    AND c.perkey = f.perkey
    AND d.classkey = e.classkey
    AND d.classkey = f.classkey
    AND d.prodkey = f.prodkey
    AND region LIKE 'West%'
    AND year = 2004
    AND class_type = 'Pkg_coffee'
GROUP BY prod_name
ORDER BY prod_name;
\end{lstlisting}

\begin{lstlisting}[language=]
PROD_NAME                      BEACHES                           CUPERTINO                        
------------------------------ --------------------------------- 
Aroma Roma                                               3483.50                           4491.00
Cafe Au Lait                                             3129.50                           4375.50
Colombiano                                               2298.25                           2653.50
Demitasse Ms                                             4529.25                           3936.50
Expresso XO                                              4132.75                           4689.25
La Antigua                                               4219.75                           2932.00
Lotta Latte                                              3468.00                           5146.00
NA Lite                                                  4771.00                           4026.00
Veracruzano                                              4443.00                           3285.00
Xalapa Lapa                                              4304.00                           5784.00
\end{lstlisting}



\end{document}
